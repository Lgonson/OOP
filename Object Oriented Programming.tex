\documentclass{article}

\usepackage{listings}
\usepackage{color}

\definecolor{dkgreen}{rgb}{0,0.6,0}
\definecolor{gray}{rgb}{0.5,0.5,0.5}
\definecolor{mauve}{rgb}{0.58,0,0.82}

\lstset{frame=tb,
	language=Python,
	aboveskip=3mm,
	belowskip=3mm,
	showstringspaces=false,
	columns=flexible,
	basicstyle={\small\ttfamily},
	numbers=none,
	numberstyle=\tiny\color{gray},
	keywordstyle=\color{blue},
	commentstyle=\color{dkgreen},
	stringstyle=\color{mauve},
	breaklines=true,
	breakatwhitespace=true,
	tabsize=3
}

\usepackage{graphicx}
\setlength{\parindent}{0pt}

\usepackage{hyperref}
\hypersetup{
	colorlinks=true,
	linkcolor=blue,
	filecolor=magenta,      
	urlcolor=cyan,
	pdftitle={Overleaf Example},
	pdfpagemode=FullScreen,
}

\urlstyle{same}



\begin{document}
\title{Object Oriented Programing with Python}
\date{05/03/2024}
\maketitle

\textbf{Tutorial URL}:
\url{https://www.youtube.com/watch?v=Ej_02ICOIgs&t=107s}

\textbf{Creating an object with a class (basic):}
\begin{lstlisting}
class Item:
	pass

item1 = 'Phone'
item1_price = 100
item1_quantity = 5
item1_price_total = item1_price * item1_quantity
\end{lstlisting}

\textbf{Creating the first method}\\
Methods are functions inside classes.

How we can go ahead and design some methods, which are going to be allowed to be executed on our instances?

The answer is, the methods should be created inside our class.

\begin{lstlisting}
	class Item:
		def calculate_total_price(self):
		pass
	
	
	item1 = Item()
	item1_price = 100
	item1_quantity = 5
	item1_price_total = item1_price * item1_quantity
\end{lstlisting}

\textbf{Self}: Python passes the object itself as the first argument everytime. So we are not allowed to create methods without the self. 

\begin{lstlisting}
class Item:
def calculate_total_price(self, x, y):  # x, and y are parameters
return x * y


item1 = Item()
item1.name = "Phone"
item1.price = 100
item1.quantity = 5
print(item1.calculate_total_price(item1.price, item1.quantity))  # When I call the method calculate_total_price, Python passes the variables as mehtods.
\end{lstlisting}

\textbf{Magic Methods}

\_\_init\_\_

\begin{lstlisting}
	class Item:
	def __init__(self):  # Python executes the init function automatically.
	
	print("I AM CREATED")
	
	
	item1 = Item()
	item1_name = "Phone"
	item1.price = 100
	item1_quantity = 5
	
	
	item1 = Item()
	item_name = "Laptop"
	item1_price = 1000
	item1_quantity = 3


Output:
	I AM CREATED
	I AM CREATED
	
\end{lstlisting}
How to avoid creating the whole attribute, adding more parameters into the class:Assign the attributes dinamicaly. 
\begin{lstlisting}
	class Item:
	def __init__(self, name):
	self.name = name
	print(f"An instance created from instance: {name}")
	
	
	item1 = Item("Phone")
	#item1.name = "Phone"
	item1.price = 100
	item1_quantity = 5
	
	item2 = Item("Laptop")
	#item2.name = "Laptop("
	item2_price = 1000
	item2.quantity = 3
	
	An instance created from instance: Phone
	An instance created from instance: Laptop
	
\end{lstlisting}


\begin{lstlisting}
	class Item:
	def __init__(self, name):
	self.name = name
	print(f"An instance created from instance: {name}")
	
	
	item1 = Item("Phone")
	#item1.name = "Phone"
	item1.price = 100
	item1_quantity = 5
	
	item2 = Item("Laptop")
	#item2.name = "Laptop("
	item2_price = 1000
	item2.quantity = 3
	
	An instance created from instance: Phone
	An instance created from instance: Laptop
	
\end{lstlisting}

	
\end{document}